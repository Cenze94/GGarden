\section{Linguaggi Utilizzati}{
	La struttura del sito è stata realizzata utilizzando il linguaggio XHTML 1.0, validato correttamente secondo gli standard del W3C. \\
	\\
	La presentazione è stata costruita in CSS, cercando di utilizzare quanto più possibile CSS2, che non valida per poche proprietà. Risulta invece valido con CSS3, secondo gli standard W3C.
	\\
	Javascript è stato utilizzato per definire funzioni di utlità alle pagine, creare contenuti dinamici (es. aggiungere campi dati nella form lato amministratore) ed effettuare controlli dinamici sui dati inseriti nelle form (pagine contattaci pannello amministratore), per nascondere e mostrare il pannello d'accesso dell'amministratore del sito e mostrare le foto nella pagina delle realizzazioni.
	\\
	La gestione dei dati è stata affidata ad XML, validato correttamente rispetto ad appositi XML-Schema. Questi ultimi definiscono i vari tag che possono comparire nei vari file .xml ed i vincoli d'unicità necessari. I file XML usati sono \textbf{database.xml}, il cui schema è \textbf{database.xsd}, e \textbf{profili.xml}, il cui schema è \textbf{profili.xsd}.\\
	I file \textbf{database.xslt} e \textbf{search.xslt} sono stati usati per generare delle pagine XHTML in base ai dati contenuti nei file XML; definiscono rispettivamente la struttura delle pagina della vendita e quella destinata a mostrare i risultati della ricerca. Non è stato usato un unico file per entrambi i casi perché ci sono delle notevoli differenze di rappresentazione, ad esempio nella vendita le piante e gli attrezzi sono divisi in due sezioni a sé, invece nella ricerca c'è un'unica sezione dove sono presenti entrambi secondo criteri diversi dalla pagina precedente.
	Vengono gestiti da PERL sia perché è necessario svolgere alcune operazioni a livello server prima di generare le pagine XHTML, sia per il fatto che il server di Tecnologie Web permette di accedere alla cartella dove sono contenuti tali file solo tramite PERL.
	\\
	Per il comportamento è stato utilizzato il linguaggio PERL, che permette di generare facilmente pagine dinamiche in XHTML. Le librerie incluse per tale linguaggio sono: strict per evitare di usare inconsapevolmente alcune espressioni del linguaggio poco sicure; warnings per visualizzare gli avvisi di possibili malfunzionamenti del codice; CGI::Carp qw(fatalsToBrowser) per mostrare nel browser gli errori dei file PERL rilevati durante la loro esecuzione; XML::LibXML per la gestione dei file XML; XML::LibXSLT per la gestione dei file XSLT; HTML::Entities per individuare e trasformare i caratteri speciali di HTML dalla rappresentazione ASCII normale, ad esempio ‘‘\&’’, nella versione codificata, ad esempio ‘‘\&amp;’’, e viceversa; CGI per recuperare i dati invitati al server tramite le chiamate GET e POST contenuti nelle form e nei link delle pagine HTML; CGI::Session per la creazione, l'eliminazione e la verifica dell'esistenza delle sessioni; infine Net::SMTP per inviare una email automatica tramite il server di Tecnologie Web.
}