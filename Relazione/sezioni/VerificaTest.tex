\section{Verifica e Test}{
	Al fine di riuscire a garantire una corretta visualizzazione del sito ed una sua fruizione da parte di un numero di browser quanto più ampio possibile, è stata verificata la validità di tutte le pagine, indipendentemente dal tipo statico o dinamico; si è infine visualizzato il sito su browser meno recenti.
	\subsection{Validazione}{
		I validatori offerti dal W3C sono stati usati ovunque fosse possibile:
		\begin{itemize}\itemsep1pt
			\item per validare le pagine HTML, anche quelle generate tramite PERL e XSLT: \url{http://validator.w3.org/};
			\item per validare i fogli di stile CSS: \url{https://jigsaw.w3.org/css-validator/};
		\end{itemize}
	}
	\subsection{Test}{
		%
	}
	%inizio sezione giusta
	\subsection{Ambiente di lavoro}{
		Abbiamo utilizzato il servizio di hosting GitHub per ospitare il nostro progetto e lavorare in contemporanea, potendo eventualmente far regredire il prodotto in caso di modifiche errate. Questo ci ha permesso di poter lavorare anche da remoto avendo sempre la versione più aggiornata a disposizione.
	}
	\subsection{Dispositivi utilizzati}{
		Oltre a testare il sito sui computer del laboratorio, per verificarne la compatibilità su un gran numero di dispositivi ci siamo affidati ai servizi offerti da \href{https://www.browserstack.com/screenshots}{BrowserStack}, correggendo di volta in volta gli errori trovati.  Di seguito l'elenco 
	}
	\subsection{Verifica del codice prodotto}{
		Per la verifica del codice HTML e CSS prodotto ci siamo affidati ai validatori messi a disposizione dal W3C.\\
		\\
		Tutto il codice XHTML prodotto è stato validato correttamente alla versione XHTML 1.0 Strict, senza errori nè warning.\\
		\\
		Abbiamo utilizzato anche \href{https://validator.w3.org/mobile-alpha/}{W3C Mobile checker} e \href{https://www.google.com/webmasters/tools/mobile-friendly/}{Google Mobile-Friendly Test} per verificare se il sito era mobile-friendly: l'esito è stato positivo.
		Il codice CSS è stato validato in versione 2, 2.1 e 3; CSS 3 ha validato correttamente i fogli di stile prodotti, senza errori, nè warning; CSS 2 e CSS 2.1 riportavano errori determinati da alcune istruzioni non pienamente compatibili, che però non ostacolano la visualizzazione o l'accessibilità del sito. Abbiamo fatto in modo di avere un sito fruibile ed utilizzabile anche in assenza di fogli di stile.\\
		\\
		Per essere certi d'aver prodotto un sito web accessibile a tutte le tipologie d'utente, sono stati usati i seguenti validatori, che effettuano un controllo run-time del sito:
		\begin{itemize}\itemsep1pt
			\item \href{https://www.totalvalidator.com/}{Total Validator};
			\item \href{http://achecker.ca/checker/index.php}{Achecker};
			\item \href{http://wave.webaim.org/}{wave.webaim.org};
		\end{itemize}
		Tutte le pagine del sito sono risultate valide.\\
		Siamo riusciti ad ottenere la certificazione WCAG 2.0 AAA, la più alta possibile. Al momento della validazione sono stati prodotti dei warnings, ma solo nella parte riguardante il contenuto. Abbiamo fatto validare le pagine anche per la Sezione 508 e la Legge Stanca, con esito positivo.
		I file XML e XSD sono stati validati con \textit{xmllint}, un programma per Ubuntu che da riga di comando valida un file XML rispetto ad uno schema XSD e fornisce tutti i problemi trovati. L'esito della validazione non ha segnalato errori.
	}
	\subsection{Conclusioni}
		Il sito, in seguito alle validazioni, si è dimostrato accessibile e compatibile con un ampio numero di browser e dispositivi.
}