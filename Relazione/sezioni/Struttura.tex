\section{Struttura}{
	La struttura del sito è stata divisa secondo le operazioni disponibili, così da semplificarne l'uso ed aiutare l'utente a trovare più facilmente le informazioni.
	\\
	Il sito è stato sviluppato in XHTML 1.0 Strict; la pagina di ricerca e quella delle vendite sono state generate tramite XSLT, modificando eventualmente il foglio di stile con i file PERL; tramite questi ultimi infine vengono mostrati i messaggi di errore in caso di dati inviati errati o di malfunzionamenti del server o del codice, le varie form di inserimento o modifica dei prodotti e l'esecuzione di queste operazioni in base ai dati inseriti dall'amministratore.
	\\
	Di seguito l'elenco delle pagine sviluppate:
	\begin{itemize}
		\item \href{http://tecnologie-web.studenti.math.unipd.it/tecweb/~pgabelli/}{public_html/index.html} In questa pagina sono state inserite le informazioni aziendali ed una breve descrizione dell'attività;
		\item \href{http://tecnologie-web.studenti.math.unipd.it/tecweb/~pgabelli/}{public_html/contattaci.html}: all'interno della quale si trova una form da cui contattare l'azienda, i numeri di telefono ed una mappa con cui raggiungere \textbf{\ggt} 
		\item \href{http://tecnologie-web.studenti.math.unipd.it/tecweb/~pgabelli/}{public_html/realizzazioni.html}: in questa pagina si trova una galleria fotografica di alcune realizzazioni di \textbf{\ggt}.
		\item \href{http://tecnologie-web.studenti.math.unipd.it/tecweb/~pgabelli/}{cgi-bin/log.cgi} 
	\end{itemize}
}
