\section{Architettura}{
	Il layout è stato strutturato allo scopo di rendere il sito fruibile indipendentemente dal dispositivo, definendo:
	\begin{itemize}\itemsep1pt
		\item layout per dispositivi desktop
		\item layout per dispositivi mobili
		\item layout di stampa
	\end{itemize} 
	
	\subsection{Progettazione layout}{
		Si è deciso di utilizzare un layout di tipo adattativo, specificando 
	}
	\subsection{Sviluppo layout}{
		Nell'immagine che segue viene mostrata la struttura data ai vari blocchi \textit{div} che compongono il sito, in cui sono contenute le informazioni divise per area tematica le informazioni.
	\\
	Il layout si adatta in base alla larghezza dello schermo, senza però porre condizioni sull'altezza, con l'obiettivo di far evolvere in altezza il sito.
		\begin{figure}[H]
			\includegraphics{\docsImg Schemasito.png}
			\caption{Schema del sito}
			\label{Schema del sito}
		\end{figure}
		
		Viene applicato un foglio di stile ai browser standard per computer desktop e portatili e fino a che la larghezza dello schermo rimane maggiore di 650 px: al di sotto si passa ad usare il CSS destinato al mobile.
		
		Analizzando il sito spostandosi dall'alto verso il basso, segue che:
		\begin{itemize}
			\item Il div \textbf{header} che informa l'utente su ciò che sta visitando; comprende:
			\begin{itemize}
				\item Titolo e logo; nella pagina relativa alle vendite è presente inoltre il pulsante d'accesso amministratore che mostra un menù a scomparsa da cui poter effettuare il login;
				\item Il div \textbf{breadcrumbs} che aiuta l'utente ad identificare la posizione in cui si trova all'interno del sito, rispetto alla homepage; al suo interno è presente anche la barra di ricerca con cui trovare prodotti o servizi all'interno del sito;
				\item Il menù utente, identificato con il div \textbf{menu}, presenta 4 scelte: "Home", "Realizzazioni", "Contattaci". Viene evidenziata la posizione corrente, rappresenta l'asse informativo di dove si può andare. 
			\end{itemize}
			\item Segue il contenuto vero e proprio della pagina, inserito nel div \textbf{content}, ha il compito di esporre le informazioni che si stanno trattando;
			\item Alla fine della pagina, il \textbf{footer}, al cui interno è presente:
			\begin{itemize}
				\item un piccolo logo dell'azienda;
				\item collegamenti alle pagine del sito;
				\item i principali riferimenti all'azienda;
				\item motto aziendale.
			\end{itemize}
		\end{itemize}
		}
	\subsection{Layout per dispositivi mobili}{
		Il layout per dispositivi mobili è stato sviluppato in modo da favorire l'incolonnamento degli elementi, rimuovendo quanto possibili margini e padding, sfruttando al meglio l'area disponibile; sono state infine ridimensionate le immagini presenti.
	}
	\subsection{Layout di stampa}{
		Nel layout di stampa sono stati tolti gli elementi che non portavano informazioni significative; i contenuti sono stati privati dei colori; è stato rimosso il menù; nella pagina delle realizzazioni è stata messa in risalto l'immagine selezionata; nella pagina relativa alla vendite si è fatto in modo da non spezzare su più pagine le informazioni relative al singolo articolo all'interno del menù.
	}
}