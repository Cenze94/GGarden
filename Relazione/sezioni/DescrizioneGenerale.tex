\section{Descrizione Generale}{
	\subsection{Il sito}{
		Il sito desidera essere una vetrina di prodotti per il giardinaggio con la possibilità di visualizzare le principali caratteristiche e le opzioni di ognuno.
		A supporto dell'utente si desidera rendere disponibili una mappa per raggiungere l'azienda, orari, e-mail e numero di telefono aziendali.
	}
	\subsection{Caratteristiche degli utenti}{
		Gli utenti del sito saranno persone interessate al giardinaggio ed al fai-da-te, in grado di utilizzare strumenti per la navigazione web. \\
		Il sito è rivolto ad un pubblico generico, all'interno del quale possiamo individuare le seguenti categorie:
		\begin{description}\itemsep1pt
			\item[Categoria di utenti:] privati e piccole aziende
			\begin{description}\itemsep1pt
				\item[Funzionalità:] Informarsi sugli attrezzi per il giardinaggio prima di procedere all'acquisto presso la sede dell'azienda.
				\item[Termini generali:] Non eccessivamente distante dal punto vendita, in un raggio di circa 70 Km.
			\end{description}
			\item[Categoria di utenti:] amministratori
			\begin{description}\itemsep1pt
				\item[Funzionalità:] area riservata da cui aggiungere, rimuovere o aggiornare prodotti disponibili;
			\end{description}
		\end{description}
	}
	\subsection{Vincoli generali}{
		\begin{itemize}\itemsep1pt
			\item Il sito dev'essere accessibile da parte di categorie d'utenti diversificate ed utilizzando dispositivi diversi compresi smartphones e tablet;
			\item Il sito dev'essere visitabile tramite i seguenti browser: 
				\begin{itemize}
					\item Firefox 3.0.1
					\item Internet Explorer dalla versione 7 alla versione 11; 
					\item Chrome 1.0.154.53
					\item Opera 9.64
					\item Safari 3
					\item Edge 13
				\end{itemize}
			\item Separazione tra struttura, presentazione, comportamento;
			\item Conformità agli standard W3C per XHTML, CSS, JS;
			\item Sito comprensibile da screen-reader.
		\end{itemize}
	}
	\subsection{Requisiti}{
		Di seguito sono presentati i requisiti emersi dall'analisi iniziale e quelli che si sono aggiunti nel corso dello svolgimento del progetto. Ciascuno è identificato da un numero progressivo per semplificarne l'individuazione successiva.\\
		\newcounter{magicrownumbers}
		\newcommand\rownumber{\stepcounter{magicrownumbers}\arabic{magicrownumbers}}
		\begin{table}[h]
			\centering
			\begin{tabular}{|p{\dimexpr 0.15\linewidth-2\tabcolsep}|p{\dimexpr 0.8\linewidth-2\tabcolsep}|}
				\hline
			 	\textbf{ID Req.} & \textbf{Descrizione}\\
				\hline
				\centering \rownumber	&	Il sito dev'essere visualizzabile sui browser elencati all'interno di "Vincoli generali"\\
				\hline
				\centering \rownumber	&	Il sito dev'essere accessibile indipendentemente dalla grandezza dello schermo del dispositivo\\
				\hline
				\centering \rownumber	&	Il sito dev'essere fruibile anche senza richiedere un foglio di stile\\
				\hline
				\centering \rownumber	&	Le figure significative dovranno essere comprensive di un attributo alt per favorire l'accesso ad utenti non vedenti\\
				\hline
				\centering \rownumber	&	Ad i tag quali <input> e <textarea> devono essere associati tabindex e accesskey\\
				\hline
				\centering \rownumber	&	Le gradazioni di colori non devono risultare sgradevoli o di intralcio a persone affette da daltonismo\\
				\hline
				\centering \rownumber	&	Il layout deve risultare fluido nel ridimensionamento del carattere tramite i tasti Ctrl + e Ctrl -\\
				\hline
				\centering \rownumber	&	Il sito deve essere validato per la parte di XHTML2.0, CSS3 e secondo gli standard WAI\\
				\hline
			\end{tabular}
			\label{tab:requisiti}
			\caption{Elenco dei requisiti}
		\end{table}
	}
}