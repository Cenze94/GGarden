\section{Linguaggi Utilizzati}{
	La struttura del sito è stata realizzata utilizzando il linguaggio XHTML 1.0, validato correttamente secondo gli standard del W3C. \\
	\\
	La presentazione è stata costruita in CSS, cercando di utilizzare quanto più possibile CSS2, che non valida per poche proprietà. Risulta invece valido con CSS3, secondo gli standard W3C.
	\\
	Javascript è stato utilizzato per definire funzioni di utlità alle pagine, creare contenuti dinamici (es. aggiungere campi dati nella form lato amministratore) ed effettuare controlli dinamici sui dati inseriti nelle form (pagine contattaci pannello amministratore), per nascondere e mostrare il pannello d'accesso dell'amministratore del sito e mostrare le foto nella pagina delle realizzazioni.
	\\
	La gestione dei dati è stata affidata ad XML, validato correttamente rispetto ad appositi XML-Schemi, forniti. Questi ultimi definiscono i vari tag che possono comparire nei vari file .xml ed i vincoli d'unicità necessari.\\
	Gli schemi \textbf{database.xslt} e \textbf{search.xslt} definiscono rispettivamente la struttura delle pagina della vendita e quella destinata a mostrare i risultati della ricerca.
	\\
	Il linguaggio XSLT è stato usato per creare dei template delle pagine riservate a mostrare i contenuti, ma è risultato più pratico lasciare a script PERL la conversione da XML ad HTML.\\
	\\
	Per il comportamento è stato utilizzato il linguaggio PERL, che permette di generare facilmente pagine dinamiche in XHTML. Le librerie incluse per tale linguaggio sono: XML::LibXML per la gestione dei file XML, XML::LibXSLT per la gestione dei file XSLT, HTML::Entities per trasformare i caratteri speciali di HTML nei corrisponenti di PERL, CGI per le sessioni e per recuperare i dati delle form HTML, infine CGI::Session per la gestione delle sessioni;
}