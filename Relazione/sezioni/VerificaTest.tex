\section{Verifica e Test}{
	Al fine di riuscire a garantire una corretta visualizzazione del sito ed una sua fruizione da parte di un numero di browser quanto più ampio possibile, è stata verificata la validità di tutte le pagine, indipendentemente dal tipo statico o dinamico; si è infine visualizzato il sito su browser meno recenti.
	\subsection{Validazione}{
		I validatori offerti dal W3C sono stati usati ovunque fosse possibile:
		\begin{itemize}\itemsep1pt
			\item per validare le pagine HTML, anche quelle generate tramite PERL e XSLT: \url{http://validator.w3.org/};
			\item per validare i fogli di stile CSS: \url{https://jigsaw.w3.org/css-validator/};
		\end{itemize}
	}
	\subsection{Test}{
		
	}
%	\subsection{Strumenti per la scrittura del codice}
	\subsection{Ambiente di lavoro}
%	\subsection{Criticit\`a librerie}
	\subsection{Test interazione utenti}
	\subsection{Dispositivi utilizzati}
	\subsection{Verifica del codice prodotto}{
		Per la verifica del codice HTML e CSS prodotto ci siamo affidati ai validatori messi a disposizione dal W3C.\\
		\\
		Il codice HTML è stato validato correttamente alla versione XHTML 1.0 Strict, senza errori nè waring.\\
		\\
		Il codice CSS è stato validato in versione 2, 2.1 e 3.\\
		CSS 3 ha validato correttamente i fogli di stile prodotti, senza errori, nè warning; CSS 2 e CSS 2.1 riportavano errori determinati da alcune istruzioni non pienamente compatibili, che però non ostacolano visualizzazione nè accessibilità del sito. \\
		\\
		Per essere certi d'aver prodotto un sito web accessibile a tutte le tipologie d'utente, sono stati usati i seguenti validatori, che effettuano un controllo run-time del sito:
		\begin{itemize}\itemsep1pt
			\item Total Validator;
			\item \url{http://achecker.ca/checker/index.php};
			\item \url{http://wave.webaim.org/};
		\end{itemize}
		Tutte le pagine del sito sono state validate.\\
		Siamo riusciti ad ottenere la certificazione WCAG 2.0 AAA, la più alta possibile. Al momento della validazione sono stati prodotti dei warnings, ma solo nella parte riguardante il contenuto.\\
		I file XML e XSD sono stati validati con:
		\begin{itemize}\itemsep1pt
		\item \url{http://www.freeformatter.com/xml-validator-xsd.html}
		\item \url{http://www.utilities-online.info/xsdvalidation/}
		\end{itemize}
		Entrambi permettono di validare singoli XML e XSD, permettono anhce la validazione incrociata per trovare eventuali errori di contenuto o di forma.
	}
	\subsection{Conclusioni}
		Il sito, in seguito alle validazioni, si è dimostrato accessibile e compatibile con un ampio numero di browser e dispositivi.
}